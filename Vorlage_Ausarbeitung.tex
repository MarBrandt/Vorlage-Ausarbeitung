\input{./latex/preamble.tex}

\newcommand*{\mytitle}{\LaTeX-Vorlage für Berichte, Bachelor- oder Masterarbeiten} % Titel
\newcommand*{\myinstitute}{Hochschue Flensburg} % Universität / Hochschule
\newcommand*{\myfaculty}{Fachbereich 1: \textit{Musterstudiengang}} % Fachbereich / Fakultät und Studiengang
\newcommand*{\myauthor}{Max Mustermann} % Autoren
\newcommand*{\myreporttype}{Abschlussarbeit} % Bericht-Typ
\newcommand*{\mygraduation}{Master of Engineering} % angestrebter Abschluss
\newcommand*{\firstexaminer}{Prof. Dr.-Ing. Max Mustermann} % Erstprüfer
\newcommand*{\secondexaminer}{Prof. Dr.-Ing. Maxima Musterfrau} % Zweitprüfer
\newcommand*{\mydate}{\today} % Datum


% ============= Dokumentbeginn =============

\begin{document}
	
% Seiten ohne Kopf- und Fußzeile sowie Seitenzahl
\pagestyle{empty}

\begin{center}
\begin{tabular}{p{\textwidth}}

\begin{center}
	\includegraphics[scale=0.35]{img/logos.jpg}
\end{center}


\\

\begin{center}
\LARGE{\textbf{
\LaTeX-Vorlage für Berichte, Bachelor- oder Masterarbeiten\\[1cm]
}}
\end{center}

\\


\begin{center}
\large{Hochschule Flensburg\\
Fachbereich 1: \textit{Musterstudiengang}\\}
\end{center}

\\\\

\begin{center}
\textbf{\Large{Abschlussarbeit}}
\end{center}


\begin{center}
zur Erlangung des akademischen Grades\\
Master of Engineering
\end{center}

\\\\

\begin{center}
vorgelegt von
\end{center}

\begin{center}
\large{\textbf{Max Mustermann}} \\
% \small{geboren am 01.01.0001 in Entenhausen}
\end{center}

\begin{center}
\large{\today}
\end{center}

\\
\\
\\

\begin{center}
\begin{tabular}{lll}
\textbf{Erstprüfer:} & & Prof. Dr.-Ing. Max Mustermann\\
\textbf{Zweitprüfer:} & & Prof. Dr.-Ing. Maxima Musterfrau\\
\end{tabular}
\end{center}

\end{tabular}
\end{center}	% Einbinden der Titelseite
\newpage 					% Um Seite nach der Titelseite einzubinden -> bei eigener Titelseite und nicht der Latex-Version erforderlich
\thispagestyle{empty}
\quad 
\newpage
\pagenumbering{Roman}
 
\cleardoubleoddpage


\chapter*{Eidesstattliche Erklärung}
Ich versichere, dass ich die vorliegende Thesis ohne fremde Hilfe selbstständig verfasst und nur die angegebenen Quellen benutzt habe.
\\~\\
Flensburg, den \today\\[.6cm]
\myauthor\\
\rule[0.5em]{20em}{0.5pt}

\input{src/Zusammenfassung}

\tableofcontents			% Inhaltsverzeichnis
\listoffigures				% Verzeichnis aller Bilder
\listoftables				% Verzeichnis aller Tabellen
\chapter*{Abkürzungsverzeichnis}
	\begin{acronym}[4GDH]	% Zur Formatierung hier die längste Abkürzung des Verzeichnisses eintragen
		\acro{4GDH}{4th Generation District Heating}
	\end{acronym}

\pagestyle{fancy}

\chapter{Einleitung}
\pagenumbering{arabic}
	\section{Hintergrund}
	
	\section{Methodik}
	
	\section{Stand der Wissenschaft}


\input{src/Grundlagen}
\input{src/Hauptteil}
\input{src/Ergebnisse}
\input{src/DiskussionderErgebnisse}

%Literaturverzeichnis
\newpage
\lhead{}
\rhead{\leftmark}
\addcontentsline{toc}{chapter}{Literaturverzeichnis}
\bibliography{LiteraturDB}

\appendix

\include{src/anhang}

\end{document}